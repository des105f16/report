% !TEX root = ../master.tex
\pdfbookmark[0]{English title page}{label:titlepage_en}
\aautitlepage{%
  \englishprojectinfo{
    C Timed Information Flow% \\
    %-- A Practical Approach%title
  }{%
    IoT Security %theme
  }{%
    Spring Semester 2016 %project period
  }{%
    DES105F16 % project group
  }{%
    %list of group members
    Mikael Elkiær Christensen\\
    Mikkel Sandø Larsen
  }{%
    %list of supervisors
    René Rydhof Hansen\\
    Mads Chr. Olesen
  }{%
    2 % number of printed copies
  }{%
    \today % date of completion
  }{%
    \url{https://github.com/des105f16/Editor/tree/d074ede4d55c6565b9d6eac651e24dff32d754cf}
  }%
}{%department and address
  \textbf{Department of Computer Science}\\
  Aalborg University\\
  \href{http://cs.aau.dk}{http://cs.aau.dk}
}{% the abstract
  This report describes the tool \thelanglong{} (\thelang), which can take new and existing C source code, allowing for labeling of security policies applied directly to the source.
  The tool can then, based on the labeling, provide feedback about any potential breaches of security as labeled information flows through the program.

  These security policy labels are based on \emph{The Decentralized Label Model} (DLM).
  The report takes important concepts of DLM, extending on these in order to provide an extended description as well as a formalization in regards to the inferrance of security policy labels.
  Additionally, an extension to the security policies are provided, by allowing the expression of time policies.
  These time policies are similarly created with a focus on their simplicity and practical applications.

  As a first step in formalizing the time policy extension, it will be shown how they can be translated into \emph{timed automata}.
  In order to compare the practical applications, the time policies will be compared with \emph{The Timed Decentralized Label Model}, which takes a more theoretical approach.
}
