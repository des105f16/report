\theoremheaderfont{\normalfont\bfseries}
\theorembodyfont{\normalfont}
\theoremstyle{break}
\def\theoremframecommand{{\color{gray!50}\vrule width 5pt \hspace{5pt}}}
\newshadedtheorem{exa}{Example}[chapter]
\newenvironment{example}[1]{%
		\begin{exa}[#1]
}{%
		\end{exa}
}

\newshadedtheorem{defini}{Definition}[chapter]
\newenvironment{definition}[1]{%
    \begin{defini}[#1]
}{%
    \end{defini}
}

\newcommand{\iVar}{\mathbf{Var}}
\newcommand{\iFun}{\mathbf{Fun}}
\newcommand{\iDecv}{\mathbf{DecV}}
\newcommand{\iDecf}{\mathbf{DecF}}
\newcommand{\iStmc}{\mathbf{StmC}}
\newcommand{\iExpd}{\mathbf{ExpD}}
\newcommand{\iLbl}{\mathbf{Lbl}}

\newcommand{\trtspc}{\hspace{2em}} %transition relation table space

\newcommand{\trule}[4]{% NAME, BOTTOM, TOP, WHERE
\begin{align*}
  \text{[#1]} \hspace{2em}  & \ifthenelse{\equal{#3}{}}{#2}{\frac{#3}{#2}} \\[1em]
                            & \parbox{\textwidth}{$#4$}
\end{align*}}

\DeclareMathOperator*{\sqcupl}{\sqcup}
