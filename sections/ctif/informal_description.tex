% !TEX root = ../../master.tex

\section{Scope}\label{ctif:scope}
We have tried to include as many C constructs as possible, in order to properly ascertain the usefulness of \thelang.
However, certain aspects of C have been either simplified or completely left out.

In the following is a crude list with what is currently supported in \thelang.
Following the list will be some elaborations where necessary.
\begin{itemize}
  \item Variables
  \item Simple types (int, bool, char, string)
  \item Structs
  \item Pointers
  \item Function prototypes, declarations, and calls
  \item Boolean and arithmetic operators
  \item \emph{if}- and \emph{while} control statements
  \item External libraries
\end{itemize}

\myparagraph{Structs and arrays}
It is possible to define structs, so that the defined type can be used similarly to any other type.
Both structs and arrays can be declared, arrays by using pointer-notation.
However, it is not possible to initialize neither struct or array.
We still have index and element expressions, with a simple implementation, which only takes into account the label of the struct/array variables.

We have chosen to leave initilization for structs and arrays out for now, as we found no proper solution for handling policies for individual fields and elements.
In \cite{myers1997} labels can be applied to structs by labeling the variable.
Additionally, fields can be labeled, resulting in a special rule for the evaluation of accessing struct fields, which is the label join of the struct label and the accessed field's label.
We, however, find it somewhat illogical to label a type, since different usage of a certain type may call for different labeling.

\myparagraph{Pointers}
Similar to both \cite{muller2015cif} and \cite{maciazek2016cbif} we allow for pointer declarations and for the two pointer operations dereference and address-of, but disallow pointer arithmetics.
Pointer arithmetics is deemed too big of a subject to handle within the scope of this report, and without proper handling it is potentially highly error-prone.\footnote{\url{https://www.securecoding.cert.org/confluence/display/c/EXP08-C.+Ensure+pointer+arithmetic+is+used+correctly}}

\myparagraph{Boolean and arithmetic operators}
The following boolean and arithmetic operators are allowed:
\[ \tk +, \tk - \text{(unary and binary)}, \tk *, \tk /, \tk{\&\&}, \tk{||}, \tk{!}, \tk * , \tk \& \]

Operator precedence has no say for our compiler, as we are only interested in what operands are used in expressions.
The actual operator precedence checks are only used for, and therefore handled by, the C compiler.

\section{Informal Description}
\thelang{} extends the C language with the ability to declare labels for values and functions.
Labels are declared between the type and the identifier of a variable, or between the return type and identifier of a function.
The syntax for labels consists of 5 components:
\begin{itemize}
  \item delimiters \dlmc{\{\{ \}\}}
  \item join operator \dlmc{;}
  \item identifiers -- e.g. \dlmc{owner}, \dlmc{reader}, or \dlmc{variable}
  \item policy operator \dlmc{->} -- e.g. \dlmc{\{\{owner->reader1, reader2\}\}}
  \item special values \dlmc{_} (underscore, for bottom label) and \dlmc{^} (caret, for top label)
\end{itemize}

Besides the actual labels we also have the 3 related constructs, which were presented in \cref{dlm:auth_and_declass,dlm:channels}:
\begin{itemize}
  \item output channel \dlmc{p1, p2 <- foo();}
  \item if acts for \dlmc{-->? p1, p2 \{ /* statements */ \}}
  \item declassify \dlmc{<|v, l|>}
\end{itemize}

The following subsections will describe how labels are used in \thelang.
The descriptions will follow the password checker example from \cref{example:sec:check_password}.
The fully labeled source can be seen in \cref{example:code:check_password-explicit}.

\lstinputlisting[float, style=dlmc, numbers=left, caption={Labelled password checker example}, label=example:code:check_password-explicit]{dlm_examples/report/check_password-explicit.ncif}

\subsection{Function declarations}
The first thing we would want to label in a program is what can be seen as the bigger components; the functions that either obtain data from outside or send data to outside the program -- our channels.
In \thelang{} we have no distinction between input channels and any other function which has a return value.
However, output channels are distinguishable by having the \dlmc{<-} construct, preceded by one or more principals (readers).
Output channel declarations are initially checked as with any other function.

For \dlmc{check_password} we have the 3 auxiliary functions, as well as the \dlmc{check_password} function itself, which we would label as so:\\
\begin{minipage}{\linewidth}
\lstinputlisting[style=dlmc, numbers=left, linerange={11-15}, firstnumber=11]{dlm_examples/report/check_password-explicit.ncif}
\end{minipage}

Here we have two principals: user (u) and password checker (pc).\footnote{We make use of the \dlmc{principal} declaration in the full example to simulate the actual obtaining of system principals.}
The user provides some login information, to which he is the owner.
The password checker is a trusted principal which has access to the user database.
The final output is sent to an output channel which is readable only be the user.

\dlmc{check_password} returns the comparison result, which is owned by and readable by the user only.
Here we also need to add labels to the parameters, otherwise we would not be able to statically check the function, as the actual labels for the parameters could differ between different calls to the function, but by supplying the labels we know that the policies will have to hold for all usage.
When supplied with a label, parameters are treated as any other variable when evaluating a function declaration.
However, if no label is supplied for a parameter, that parameter is recognized as labeled with a constant label.

\subsection{Variable declarations}
The labeling of variable declarations is done mainly to oblige the return values of functions and input, given by either parameters or auxiliary function calls.
This is also why most of these labels are easier to infer, and can therefore often be left out, as we shall see later in this section.

For \dlmc{check_password} we have the following labeling of variables:\\
\begin{minipage}{\linewidth}
\lstinputlisting[style=dlmc, numbers=left, linerange={16-19}, firstnumber=16]{dlm_examples/report/check_password-explicit.ncif}
\end{minipage}

The labeling applied here is done so that it obliges the different inputs to the method.
The most noteworthy here is the labeling of \dlmc{match}, as it through its assignment within both the \dlmc{while} and \dlmc{if} blocks will implicitly obtain knowledge of all other labeled values.
It is therefore labeled with the join of all those labels: \dlmc{\{\{u->u;pc->\}\}} so that it is at most as strict as all.

\subsection{If-acts-for and declassification}\label{ctif:informal:ifactsfor_declassify}
Going into \dlmc{check_password} we have data that is privy to the user as well as data that is not.
So before we can return the result to the user, we need to explicitly declare that in this particular case it is an intentional leak of the password checker's data (the user db).

This is done with the declassification delimiter \dlmc{<| |>} which takes two arguments, an expression to be declassified and the new label.
For \dlmc{check_password} we need to declassify and return \dlmc{match}:\\
\begin{minipage}{\linewidth}
\lstinputlisting[style=dlmc, numbers=left, linerange={28-30}, firstnumber=28]{dlm_examples/report/check_password-explicit.ncif}
\end{minipage}

As can be seen we also need to obtain the proper authority before such a declassification is possible.
This is done by using the \emph{if-acts-for} statement, identified by the \dlmc{-->?} syntax, which is preceded by type of authority and followed by principals to acquire authority for.
During compile-time checking we do not differ between the special keywords \dlmc{this} and \dlmc{caller}.
After obtaining the proper effective authority, we need to clear the return value's label of the policy of \dlmc{pc}, as explained in the beginning of this subsection.

It is possible to leave out the explicit label of the declassification, as we will see in the following subsection.

\subsection{Inference}
As we have argued repeatedly throughout this report, having to declare labels for every value or function can be time-consuming and can cause a clutter in a program.
Therefore we can rely on inference so that many labels can be omitted.
In \cite{myers1997} it is only described what leaving out labels for variables and parameters mean, and not functions.
Therefore we have reasoned about the implications of leaving out the function label and come up with a reasonable solution, so that we can leave out those labels and still expect proper results.

As we have seen, explicit labels can be applied to the three types of declarations: function, parameter, and variable, as well as in the declassification expression.
Here we will explain what it means to leave out any of those labels in a program.
An overview of the different rules that apply are given here:
\begin{table}[H]
  \begin{tabularx}{\textwidth}{|l|X|}
    \hline
    \textbf{Construct} & \textbf{Default label} \\ \hline \hline
    Variable & Variable label \\ \hline
    Declassification & Variable label \\ \hline
    Function & The join of all parameter labels \\ \hline
    Parameter & Constant label \\ \hline
  \end{tabularx}
  \caption{Default labels for unlabeled constructs}
  \label{informal:table:default_labels}
\end{table}

As described in \cref{dlm:inf:types}, we have the concepts of \emph{variable labels} and \emph{constant labels}.
Variable labels are the main enablers of the inference algorithm, as it allows for implicitly labeling certain values that in themselves do not need a label, but are instead related to other values that do.
Constant labels, in a similar fashion, allow for the abstraction of only applying labels where it is critical, allowing for some implicity when defining the security of functions.
We have updated the password checker example (see \cref{example:code:check_password-implicit}), removing all explicitly declared labels that instead can be inferred.
Similarly, the calculate bill example has been labeled as well (see \cref{example:code:calculate_bill-implicit}).

\lstinputlisting[float, style=dlmc, numbers=left, caption={Labeled password checker example -- with default labels}, label=example:code:check_password-implicit]{dlm_examples/report/check_password-implicit.ncif}

\lstinputlisting[float, style=dlmc, numbers=left, caption={Labeled smart meter bill calculation example -- with default labels}, label=example:code:calculate_bill-implicit]{dlm_examples/report/calculate_bill-implicit.ncif}

\myparagraph{Variable}
As explained above, by removing the explicitly declared label for a variable, we instead rely on the inference algorithm to satisfy any constraints.
Since we have explicitly declared the labels for all our channels and \dlmc{check_password}, we do not need to explicitly label all our variables as well.
This results us in being able to leave out all label declarations for variables.

\myparagraph{Declassification}
Similar to variables, we can leave out the explicitly declared label for the declassification expression.
This creates a variable label for that expression, so that we can rely on inference in a similar fashion as with any other value labeled with a variable label.

\myparagraph{Functions and parameters}
If no explicitly declared label is found for a function declaration, the label defaults to a join of all parameter labels ($\bot$ if the function has no parameters).
This is based on an assumption about the minimum security needed for a function, in that whatever a function returns can be related to at most whatever arguments are given.
This is useful for declaring auxiliary functions, without having to explicitly declare the security implications of such a function.

For our password checker we could declare an auxiliary function \dlmc{strcicmp}, for doing a case-insensitive check of usernames, with the following prototype:\\
\begin{minipage}{\linewidth}
  \centering
  \dlmc{bool strcicmp(char *a, char *b);}
\end{minipage}
\dlmc{stricmp} would then have the label \dlmc{\{\{a;b\}}}, per our default label for function declarations, and any evaluation of calls to \dlmc{stricmp} would result in a label which is the join of all its arguments.
This is the default behaviour of library functions, which can be overridden by declaring a labeled prototype with the corresponding library function signature.

Extending on the concept of referencing parameter labels in function return labels, we can choose which parameters to reference.
If we take the function \dlmc{stricmp} and want to extend it with a maximum amount of characters to compare before failing, we could have that its signature was:\\
\begin{minipage}{\linewidth}
  \centering
  \dlmc{bool \{\{a;b\}\} stricmp(char *a, char *b, int max_length);}
\end{minipage}
Similar to the previous example, we have that any calls to \dlmc{stricmp} will be evaluated to the join of its arguments, but this time only the arguments \dlmc{a} and \dlmc{b}, thus any security applied to \dlmc{max_length} will not affect the label of the return value.
