% !TEX root = ../../master.tex

In this chapter we will present the \thelanglong{} language, hereafter referred to simply as \thelang.
However, this chapter will only show how to apply ordinary labels, as those presented in the previous chapter.
The following chapter will go into the details of time policies.

\thelang{} is based upon C99 and will extend the syntax of C with the concept of labels, as discussed in the previous chapter.
This fact gives us two nice properties:
\begin{enumerate}
  \item A C99 program is compilable with the \thelang{} compiler.
  \item An unlabeled \thelang{} program can be checked by a C99 compiler.
\end{enumerate}
The first property allow for incrementally adding labels to old source code.
This becomes even more palpable when also having the ability of using the \thelang{} compiler to infer labels.
The second property frees the \thelang{} compiler from handling many of the same things that could as easily be handled by already existing C99 compilers.

The rest of this chapter will be organized as follows.
Firstly, we will present the scope under which we have limited our implementation.
Then we will informally present \thelang~ without going into much details about the implementation, but rather focus on usage.
Lastly we will present the specifics of inferring labels, as described in \Cref{dlm:inferring_labels}.
