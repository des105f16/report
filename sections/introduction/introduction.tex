% !TEX root = ../../master.tex

\mikaelin{Bite. Mere blød introduktion.}

In this chapter we will present the needed background information which serves as the base for our project.
We will, based on our pre-specialization \cite{prespecialization}, introduce some of the security problems found in the Internet-of-Things (IoT).
Specifically, we will look into privacy issues associated with IoT.
We will also discuss the possibility of handling some of these privacy issues with time-based constraints.
Additionally, we will discuss some problems related to the development of secure software, which will be related to our solution.
Following the description of our solution, we will present some related work.
Finally, we will present two code examples, which will be used throughout the report, to serve as IoT use-cases.

The remainder of the report is structured as follows.
In \cref{dlm} we will provide the necessary background information for understanding the first part of our solution.
Then we will start presenting our solution in \cref{ctif}, by providing an informal description, followed by a formal definition of how labels are extracted and checked.
Then in \cref{time} we will introduce another addition to our solution, by extending the former with time policies, which we will compare to existing theory.
Finally, we will conclude in \cref{conclusion} along with a description of possible future works.\mikael{Mere flydende sprog, mindre ``we will''.}
