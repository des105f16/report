% !TEX root = ../../master.tex

\section{Related work}
Despite DLM having been around for near two decades now \cite{myers1997}, not much work has been put into putting the theory to use.
One exception is the \emph{Java Information Flow} (JIF)\footnote{\url{https://www.cs.cornell.edu/jif/}} project, a project related to the original authors.
However, by using JIF, and thereby Java and a Java Virtual Machine (JVM), one is somewhat limited in actual applications, especially in regards to IoT devices.
This is why we explore the possibility of applying DLM to C.

Two recent projects have also explored this; \emph{C Information Flow} (CIF) \cite{muller2015cif} and \emph{Content-Based Information Flow Control} (CBIF) \cite{maciazek2016cbif}.
These projects are very similar and demonstrate some of the properties that we seek to develop as well.
All three extend C with DLM constructs, making it possible to annotate code with security policies, and statically check that policies are not violated throughout program flow.
None of the three discuss the run-time concepts of DLM.

In CIF, a simple approach has been chosen, having only the bare essentials needed for static checking.
Besides providing static checking of DLM policies, CIF also outputs information flow graphs, providing a visual overview of the flow for a program.

In CBIF, a very different approach is used, focusing on extending the syntax with new constructs, so that model checking can be applied.
CBIF is also very domain-specific, having a focus on avionics-related use.

Not much is said about inferring labels in neither CIF nor CBIF, which is why CTIF will explore this further.
Additionally, CTIF will explore the possibility of extending the security policies with time constraints, adding further possibilities for modeling security of programs.
