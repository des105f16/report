% !TEX root = ../../master.tex

\section{Related work}
Despite DLM having been around for near two decades now \cite{myers1997}, not much work have been put into putting the theory into use.
One exception is of course \emph{Java Information Flow} (JIF)\footnote{\url{https://www.cs.cornell.edu/jif/}}, a project related to the original authors.
However, by using JIF, and thereby Java and a Java Virtual Machine (JVM), one is somewhat limited in actual applications.
This is why we want to explore the possibility of applying DLM to C.

Two recent projects have also explored this, \emph{C Information Flow} (CIF) \cite{muller2015cif} and \emph{Content-Based Information Flow Control} (CBIF) \cite{maciazek2016cbif}.
The premise for both projects, and for our project as well, is very similar.
All extend C with DLM constructs, making it possible to annotate code with security policies, and statically check that policies are not violated throughout program flow.
None spend any time on run-time related concepts.

In CIF, a simple approach has been chosen, having only the bare essentials needed for static checking.
Besides providing static checking of DLM policies, CIF also outputs information flow graphs, providing a visual overview of the flow for a program.

In CBIF, a very different approach is used, focusing on extending the syntax with new constructs, so that model checking can be applied.
CBIF is also very domain-specific, having a focus on avionics-related use.

Little is said inferring labels in neither CIF nor CBIF, which is why CTIF will explore this further.
Additionally, CTIF will explore the possibility of extending the security policies with time constraints, adding further possibilities for modeling security for programs.
