% !TEX root = ../master.tex

\section{\textbf{P}eriod, \textbf{I}nterval, \textbf{C}ount}
We expand on the concept of labels using a set of properties that describe rules for when, and how often, data can be read.
When representing these new types of constraints we have chosen a representation similar to that of labels in DLM.
This allows for the expansion to be easily understood by anyone familiar with DLM.
This is, in part, due to the fact that the operations we perform on labels (join, meet and testing for restrictiveness) have a similar definition for time.

The expansion is divided into three components:
\begin{itemize}
  \item \textbf{Period} represents a period in time, such as 08:00-09:00.
  Formally we represent this as the pair $(p_s, p_e)$ describing when a period starts and ends.
  A period cannot specify rules for particular dates or days of the week.

  \item \textbf{Interval} represents how often data can be read.
  The interval can be described as a combination of milliseconds, seconds, minutes, hours and days.
  For instance $10m30s$ represents a 10 minutes and 30 seconds long interval.

  \item \textbf{Count} represents the number of times data can be read.
\end{itemize}

\begin{definition}{Time policy}
  A set of a period, an interval and a count is called a \textit{time policy}.
  A time policy has the following formal representation: $$(p_s, p_e, i, c)$$

  Data can only be accessed in the timespan $[p_s; p_e[$.
  Data can only be accessed $c$ times per interval $i$.
  A label can contain 0 or more time policies.
\end{definition}

Syntactically it is not required that each component of a time policy is specified.
\Cref{time:implicit} describes definitions for all time policies in which one or more components have not been defined.
In these definitions the period 00:00$\rightarrow$24:00 is meant to represent the full 24 hours of a day.
\begin{align}
  () &= (\text{00:00}, \text{24:00}, 1\texttt{d}, \infty) \nonumber \\
  (p_s, p_e) &= (p_s, p_e, 1\texttt{d}, \infty) \nonumber \\
  (i) &= (\text{00:00}, \text{24:00}, i, 1) \nonumber \\
  (c) &= (\text{00:00}, \text{24:00}, 1\text{d}, c) \label{time:implicit} \\
  (p_s, p_e, i) &= (p_s, p_e, i, 1) \nonumber \\
  (p_s, p_e, c) &= (p_s, p_e, 1\texttt{d}, c) \nonumber \\
  (i, c) &= (\text{00:00}, \text{24:00}, i, c) \nonumber
\end{align}

Below we provide a definition for the join and meet operations in terms of their application to time policies.
The definitions ensure that the join of two labels time policies generates the least restrictive set of time policies that is as restrictive as both labels time policies.
A similar inverse definition is applied to the meet operation.

least restrictive that is as restrictive as both operands

\begin{definition}{Time join}
  \begin{align*}
    periods(L1 \sqcup L2) = periods(L1) \cap periods(L2)\\
    \forall p \in periods(L1 \sqcup L2), interval(L1 \sqcup L2, p) = \max(interval(L1, p), interval(L2, p))
  \end{align*}
\end{definition}
