% !TEX root = ../../master.tex

\newcommand{\tat}{\;@\;} % timed at (@)
\newcommand{\tdor}{\;||\;} % timed dlm or (||)
\newcommand{\tdand}{\;\&\&\;} % timed dlm and (&&)

\section{The Timed Decentralized Label Model}
The Timed Decentralized Label Model (TDLM) \cite{pedersen2015} is an extension to the original DLM.
TDLM extends on the security labels of DLM with the addition of time policies.
We are interested in is what kind of policies can be expressed using TDLM.
We will first briefly explore its usage and afterwards we will compare it to \thelang.

\subsection{Usage}
The overall idea is that \emph{clock comparison} rules can be added to the policies of labels.
These rules are logical comparisons of some \emph{clock variables}, or constant values, which must hold before data can be read by the concerned principal.

\subsubsection{Clock comparisons}
Clock comparisons can be added to any principal in a policy and time policies may differ between principals.
If a clock comparison is added to the \emph{owner} of a policy, it applies to all \emph{readers} in that policy.
In the following label, a simple time restriction, containing a clock comparison, is added to the read rights of principal $r_1$:
  \[ \{ o : r_1(x > 500), r_2, r_3 \} \]
The policy simply states that the principal $r_1$ is not allowed to read the attached value until the clock variable $x$ is greater than 500 ms.
There are no \emph{clock variable parameters} attached to $x$ (see below), so no reset rules have been defined, meaning that once 500 ms has passed $r_1$ may read infinitely many times, or until $x$ is reset elsewhere.
The other readers $r_2$ and $r_3$ have no time restrictions.

\subsubsection{Upper limit and reset value}
Extending on the previous example, we now have that $x$ includes some simple reset rules:
  \[ \{ o : r_1(x[1000;0] > 500), r_2, r_3 \} \]
Here we have specified that once the clock reaches an \emph{upper limit}, in this case 1000, it is reset to its \emph{reset value} of 0.
Now, $r_1$ has a 500 ms window for every second in which it can read the attached value.

\subsubsection{Reset events}
Alternatively, and not excluding the use of the previous reset method, we can add reset events to a clock variable:
  \[ \{ o : r_1(x[?xreset] > 500)[*xreset], r_2, r_3 \} \]
Here we have substituted the upper limit with a reset event $xreset$, which will reset clock variable $x$ whenever the event is triggered.\footnote{Note that there is an implicit reset value of 0 if no value is given.}
$?xreset$ names a reset event name for $x$, while $*xreset$ triggers the event $xreset$ any time $r_1$ reads the attacked value.
Now, instead of resetting every 1 second, $x$ is reset to 0 any time that $r_1$ reads the attached value, effectively limiting $r_1$ to wait at least 500 ms between each read.

\subsection{Comparison}
We now proceed to compare the time constraints expressed in TDLM and \thelang.
We note that a major difference between the two models is that TDLM supports write policies and that \thelang{} does not.
As described in \cref{dlm:policies} the concepts of read and write policies are similar, and we expect that any actual application of \thelang{} would require an extension to support write policies.
Thus we consider a comparison of read policies to be a fair comparison of the two models.

Naturally we identify that TDLM supports some complex time policies that cannot be expressed in \thelang.
However which policies that includes is not apparent.

We consider this the main strength of the time policies defined by \thelang.
Due to the simplicity of the time policies we are able to clearly express properties of \thelang{}s time constraints.
One such property is the ability to define a count in a time policy, which is not directly possible to do in TDLM.
The fact that this can be expressed in such understandable terms demonstrates the practical nature of \thelang.

From \cite{pedersen2015} we have a policy attached to smart meter consumption data.
We have removed write policies from it, to get the policy below:

\[ \{s_i : u_i, e_j (x[?reset : 1] > 90)[*reset] \} \]

Here we have a smart meter $s_i$ as owner, allowing for user $u_i$ and electrical company $e_j$ as readers.
The electrical company is associated with a time policy, determining when it can perform reads.
From reading the policy it is not clear what the desired effect of that policy is.
This makes it hard for the programmer to reason about whether or not he is using the \emph{correct} policy.

The policy ensures that the electrical company can only read consumption data once every 90 days.
It provides no restrictions on other readers.
In \thelang{} we can express a similar policy:

\begin{center}
\dlmc{\{\{s->u, e @ u: 1s; 90d \}\}}
\end{center}

This policy is much clearer in the restrictions it expresses.
Because of this \thelang{} policies can be read, left to right, directly from code.
This makes it very easy for a programmer to reason about the policy.
The above example is read as:

\begin{adjustwidth}{1cm}{1cm}
\begin{center}
\textit{The data is owned by the smart meter }\dlmc{s}\textit{, allowing reads }\dlmc{->}\textit{ by the user }\dlmc{u}\textit{ and the electrical company }\dlmc{e}\textit{.
With regards to time }\dlmc{@}\textit{, the user }\dlmc{u}\textit{ is allowed to read at most once per second }\dlmc{1s}\textit{, while everyone else at most once per 90 days }\dlmc{90d}\textit{.}
\end{center}
\end{adjustwidth}

In the TDLM policy we have a lot of complex concepts, arriving from its theory-near approach.
The syntax is near its underlying driver: the timed automata, as can be seen from the use of an actual clock comparison and the reset event and value.

We have however opted for abstracting away from these more complex concepts, such that the programmer need only focus on describing his policy in real-world terms.
