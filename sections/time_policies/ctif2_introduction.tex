% !TEX root = ../master.tex

In this chapter we will extend our previous description of \thelang{} with the use of time policies.
The same concepts as previous still hold, as we want a simple way of declaring time policies to certain values in a program, in order to restrict the access to these values.

We will make a proposal for a practical extension, of the previously discussed security labels, using time policies.
After providing a description of these new time policies, we show how they can be translated into timed automata, so that relevant theory may still apply.

Finally we compare the time policies of \thelang{} with TDLM \cite{pedersen2015}.
The two projects provide means of specifying time policies, though through different approaches.
Where \thelang{} was constructed from a practical and directly applicable perspective, TDLM tales a more formal approach.
