% !TEX root = ../../master.tex

\section{Extending the security model}
% extend concept of output channels with ensuring that time policies are followed
% current authority dictates whether a function call to a time-labeled value can be made or not
% simple, yet expressive, time labels attached to readers
% in order to check compile time:
% - special constructs @ and @?
% - limited to return labels and function root scope variables (if time-labeled and not handled in the function)
Extending on the concept of controlling how information should exit the system, as we did previously using channels, we want to add time constraints to the security model.
As we have different principals in our system, which have different requirements/rights, we want to be able to express multiple time policies which apply to different principals.
To do this we extend the previously defined syntax for labels and include some language constructs that simplify how to work with time considerations.
This language feature is not considered an extension of the labels defined by DLM.
One of the reasons for this is that statically determining which time policy is the most restrictive and how to join two policies together present some difficulties.
In \cref{time:inference} we will go into details about these challenges.

Even though the time policies do not fit in the same model as the labels defined by DLM, we have chosen to provide a shared syntax for both types of security policies.
Grouping all the security policies together like this, will make it easier for the programmer to identify which policies apply in a specific context.

\subsection{Expressiveness}\label{time:expressiveness}
We expand on the concept of labels using a set of properties that describe rules for when, and how often, data can be read.
We want our time policies to be expressive, so that we can cover a wide variety of cases.
We therefore introduce three types of time constituents, where any combination of these will form a time policy.

In the following, we will consider time values as positive integers representing milliseconds.
For simplicity, we will use the time postfixes: $\mathbb{T} = \{ h, m, s, ms \}$, which are to be seen as factors to be applied to times.
For a value $n \in \mathbb{N}$ followed by a postfix $T \in \mathbb{T}$, we have that:
\[nT = \begin{cases}
  n \times 60 \times 60 \times 1000 & \quad\text{if } T = h \\
  n \times 60 \times 1000 & \quad\text{if } T = m \\
  n \times 1000 & \quad\text{if } T = s \\
  n & \quad\text{otherwise}
\end{cases}\]
\myparagraph{Period}
Period represents a start- and end-time, where access is only permitted within that period.
E.g. \dlmc{\{\{u->u, ec@09:00-10:00\}\}} signifies that data can only be accessed when the time is between 09:00 and 10:00 in the morning.

Formally we represent a period as the pair $(p_s, p_e)$ describing when a period starts (inclusive) and ends (exclusive).
A period cannot specify rules for particular dates or days of the week.

\myparagraph{Interval}
Interval is a constant value indicating the minimum amount of time there should pass between each access.
The interval can be described as a combination of milliseconds, seconds, minutes, hours and days.
E.g. \dlmc{\{\{u->u, ec@10m30s\}\}} requires a minimum of 10 minutes and 30 seconds between each access.

\myparagraph{Count}
Count is in itself without meaning, as it says how many consecutive accesses within the given period and/or interval are allowed.
However, due to default values some meaning may apply to a case such as \dlmc{\{\{u->u, ec@10m*5\}\}}, where in this case 5 consecutive accesses are allowed in a 10 minute interval.

\myparagraph{Combined time policies}
It is possible to combine two or all three of the time constituents.
The count component even requires such a combination.
Count should be considered a ``\textit{number of reads in some period or interval}''.
If an interval is declared, count specifies the number of reads in that interval (defaulting to 1).
If an interval is not declared but a period is, count specifies the number of reads in that period (defaulting to $\infty$).

\begin{example}{Time policies}
  The \dlmc{\{\{pc->u@10m*3\}\}} policy, indicates that access is only allowed access thrice per 10 minutes.
  Another way of explaining it; after the initial read another two reads \emph{can} follow, either way it is reset 10 minutes after the initial read.

  Another example \dlmc{\{\{u->u, ec @ u:00:00-24:00 30m; 00:00-09:00\}\}}, allows \dlmc{u} unlimited access.
  Everyone else are however restricted to reading between midnight and 9 in the morning, and with a minimum of 30 minutes passing between each access.
\end{example}

From the above examples we have that a label can be extended with time policies using a \dlmc{@} followed by a list of time policies.
All except the last policy are preceded with the name of the principal to whom the policy applies.
The last policy is not associated with a principal and serves as the default time policy.
In \cref{time:authority} we describe how a policy is selected from a such a declaration of policies.

\subsection{Applying time policies to the examples}
With the language constructs described in the previous section, we will now revisit our two code examples.
In the following we will discuss how time policies can be added to the password checker and the bill calculator (see \cref{example:code:check_password-implicit,example:code:calculate_bill-implicit} on pages \pageref{example:code:check_password-implicit} and \pageref{example:code:calculate_bill-implicit} respectively).

As with labels we will only apply time policies where they directly apply to data sources.
The programmer should not be forced to declare additional policies in order to satisfy certain constraints that the compiler is trying to enforce.
In \cref{time:inference} we describe difficulties in using inference for time policies and in \cref{time:constructs} we define a set of constructs that the programmer can use to define how his time policies should be handled.

\myparagraph{Password checker}
In the password checker we would like to restrict how often the collection of user information \dlmc{get_users} can be accessed.
To achieve this effect we update the function declaration:

\begin{lstlisting}[style=dlmc]
  user_info {{pc-> @ 10m * 3}} *get_users(){};
\end{lstlisting}

Access to the data will then be restricted to three times every ten minutes, rendering brute force attacks very slow.
Unfortunately this time policy is shared across all principals, meaning that all principals/users will have to share the three accesses every ten minutes.
A simple solution to this is to define a time policy for each principal, and possibly the same policy for all.

This does not provide a dynamic solution when the number of principals is not known in advance.
But as \thelang{} was primarily developed for smaller embedded devices it can be expected that the set of principals will not be expanded dynamically.

\myparagraph{Bill calculation}
In the bill calculator we would like to restrict how often different principals are allowed to retrieve the consumption data of a users smart meter.
To do this we add a time policy to the \dlmc{get_latest_usage} function:

\begin{lstlisting}[style=dlmc]
  usage {{u->u, ec @ u: 1s, 00:00-01:00 14d}} *get_latest_usage(){};
\end{lstlisting}

Here we specify that the user is allowed to read his usage data every second, but that everyone else can only read once every other week and only during the first hour of the day.
The granularity of these policies could naturally be altered if so needed, but this policy demonstrates how easy it is to define a timed privacy policy.
The user will be able to closely watch his consumption data, while the electrical company will retrieve billable consumption data every other week.

It should be noted that in order to make this example work, the \dlmc{get_latest_usage} function would have to be maintain a list of what data each principal has received.
Though this could be handled strictly using static evaluation (using the acts for construct), it would prove a more dynamic solution to use some runtime information about which principal is executing the function.

\myparagraph{Runtime issues}
Having expanded the examples with time policies, we have noted that there are some difficulties with providing principal-specific feedback from functions.
If the set of principals is fixed (which might often be the case for embedded devices) the issues can be handled at compile time.
Should we however desire a dynamic solution for principals we would require specific runtime handling of principals.
This has not been part of \thelang{}.

\subsection{Inference of time policies}\label{time:inference}
Having described how label information is propagated from declaration to dependencies we effectively allow the programmer to specify policies only where they make some intuitive sense.
The many benefits of this has already been described and is due to the use of the inference algorithm.
We would like for this algorithm to infer time policies in the same fashion as it infers owner/reader policies.
To do that we would have to expand the label model with time policies.

Time policies could be defined as a tuple $(p_s, p_e, i, c)$, describing the period, interval and count of the policy.
Part of a label would then involve a function that defines the time policy of each principal.

In the following we will use various tuple sizes for representing time policies, depending on which components are relevant to the context.
A tuple that does not contain all four time policy components is meant to represent only a part of a policy.
Which part will be clear from the use of $p$, $i$ and $c$ in each context.

Examining the algorithm (page \pageref{dlm:inf:algorithm}) we see that in order to infer time policies we must be able to define the $\sqsubseteq$ and $\sqcap$ operations for these time policies.
Additionally we would want to define $\sqcup$ for a complete lattice of time policies.

\subsubsection{Time policy restriction}\label{time:restriction}
We will start with a definition for whether a time policy $(p_{s1}, p_{e1}, i_1, c_1)$ is no more restrictive than another $(p_{s2}, p_{e2}, i_2, c_2)$.
Defining this operation for any single time policy component is quite simple:
\begin{itemize}
  \item If a period contains another period it is no more restrictive than that period; \\
  $(p_{s1}, p_{e1}) \sqsubseteq (p_{s2}, p_{e2}) \text{ if } [p_{s1}; p_{e1}[ \; \supseteq [p_{s2}; p_{e2}[$
  \item A time interval is no more restrictive than a longer time interval; \\
  $i_1 \sqsubseteq i_2 \text{ if } i_1 \leq i_2$
  \item A number of allowed counts is no more restrictive than a lower number of counts;
  $c_1 \sqsubseteq c_2 \text{ if } c_1 \geq c_2$
\end{itemize}

When comparing two full time policies, we can start by comparing the time periods.
If we find that a policy is more restrictive here, we say that the entire policy is more restrictive.
If not, we will look at the interval and count component.
We make this distinction, as the period component specifies at which point in time the remaining components apply.

When comparing the interval and count the process is not as obvious, as we present three options for how these two components could be compared:

\begin{enumerate}
  \item Determine if interval is more restrictive and if not, determine count.
  \item Determine if count is more restrictive and if not, determine interval.
  \item Compare the two policies by the number of reads allowed per time unit; \\
  $(i_1, c_1) \sqsubseteq (i_2, c_2) \text{ if } \displaystyle\frac{c_1}{i_1} \geq \frac{c_2}{i_2}$
\end{enumerate}

We do not consider either of these policies to be the \textit{correct} one.
It might be tempting to consider the latter as the best choice since it simultaneously considers both interval and count.
It does however not encompass all considerations.
Below follows an example that illustrates the difficulty associated with selecting a definition:

\begin{example}{Comparing time policies}
  Consider the two time policies $t_1$ and $t_2$, declared as \dlmc{10m*10} and \dlmc{5m*5}.

  If we apply the first definition, we note that \dlmc{10m} is more restrictive than \dlmc{5m} and so $t_1 \sqsubseteq t_2$ must be false.
  As \dlmc{*10} is less restrictive than \dlmc{*5}, the second definition yields the same result.
  Finally using the third definition $t_1 \sqsubseteq t_2$ must be true, as each policies allows for one read per minute.

  If we flip the operands to $t_2 \sqsubseteq t_1$ the definitions yields true, true and true.
  Thus the final definition is not able to establish an order for the two policies.

  Having the freedom to do 10 reads at any time in 10 minutes does however provide some additional freedom over two times 5 reads in 5 minutes; the option to freely determine when the 10 reads are used.
\end{example}

We will not go into further details about selecting an appropriate definition.
Selecting which to use could be left to the programmer when applying time policies to his application.

\subsubsection{Combining time policies}
Having an approximate definition for the $\sqsubseteq$ operation on time policies we will try to define a matching definition for joins and meets of time policies.
As one is the inverse of the other we will only consider meets of policies.
This fills our requirement for the inference algorithm.
With a definition for meet it should be simple to also provide a definition for joins.
The meet of two time policies should be:
\begin{center}
  \textit{The most restrictive policy that is no more restrictive than either operand.}
\end{center}

As above we start by only considering time policy periods, and provide the following definition:
$$(p_{s1}, p_{e1}) \sqcap (p_{s2}, p_{e2}) = (\text{max}(p_{s1}, p_{s2}), \text{min}(p_{e1}, p_{e2})$$

Again we note that time periods yields a simple definition for our operations.
However when we examine intervals and count we run into difficulties.
As with $\sqsubseteq$ we could opt for a per-component evaluation of policies.
Thus we would use the least restrictive interval and count in our resulting policy.
We could also employ the \textit{reads per time unit} definition and use some \textit{reasonable} interval and count for the resulting policy.

Thus is it possible to define some set of operations for time policy inference.
Unfortunately, given the example below, the result is of little use to us.

\begin{example}{Issues with inference}
  Consider two slots with the time policies \dlmc{10m*10} and \dlmc{5m*5}.
  A natural use of inference is to store the sum of the two slots in a third slot, yielding a new time policy for that slot.
  Reading the value from this slot does not require time policies, as it will not be updated with a new value until the two original slots are updated.
  And these are protected by time policies.
\end{example}

The issue we encounter has to do with the nature of time policies as opposed to labels as we know them from DLM.
Labels specify what we are allowed to do, but time policies specify only when we can do it.

This is quite a different concept to handle, as understanding when something can happen should be evaluated at runtime, whereas our handling of labels and the tool we provide is based on static evaluation.
We note that labels in DLM are idempotent, as meeting or joining any label $L$ with itself results in that same label.
As this is one of the properties of a lattice, we would require our time policies to have this property as well.
However if we chose to read from the same slot twice we would have used two reads and thus the resulting label could not be the same as that of the slot.

From this it is clear that we can not provide inference of time policies in the same way that we do for labels.
Though at the same time it is also clear that we do not stand to gain much from time policy inference, as demonstrated by the example above.
Instead we have chosen to introduce certain language constructs that will allow time policies to be evaluated statically.

\subsection{Time constructs}\label{time:constructs}
When calling a function that has a time policy we could, at runtime, be in one of two different states.
Either we are in a state where the function can not be called or in one where it can.

If the function can be called we would like to do so in as normal a fashion as possible.
If the function can not be called we would require some means for how to handle a function anyway.
To satisfy these considerations we introduce the following two language constructs, specific to time policies:

\begin{itemize}
  \item if-can-call \dlmc{@?}
  \item await \dlmc{@}
\end{itemize}

The two constructs represent run-time checks and are implemented as expressions with the same precedence as function call\footnote{\url{http://en.cppreference.com/w/c/language/operator_precedence}}.
The \dlmc{@?} construct must precede a function identifier, and returns a boolean indicating whether that function is currently callable or not, as per the time constraints.
The \dlmc{@} construct is used to extent function calls. Placing \dlmc{@} before a function call, forces the program to busy-wait until the function is callable and then calls the function.

Using these constructs we can perform a check similar to a return-check where each branch of a functions control-flow will be evaluated using these constructs.
They could for instance be used in a conditional expression:
\begin{lstlisting}[style=dlmc]
  if (@?foo) {
    foo(); // Valid, due to above check.
    foo(); // Invalid, due to above call of same function.
  }

  if (@?foo) {
    foo(); // Valid, due to above check.
    @foo(); // Valid, will busy-wait if needed.
  }

  if (@?foo) {
    @foo(); // Valid, but consumes check.
    foo(); // Invalid, due to above call of same function.
  }
\end{lstlisting}
Additionally we can include checks in the condition of a while loop, letting it continuously call a function until its time policy no longer allows it.
We can also perform one-line checks and calls to evaluate the value returned from a function:
\begin{lstlisting}[style=dlmc]
  if (@?foo && foo() > 10) {
    // The value was read and it was greater than 10.
  }
  else {
    // The value could not be read, or was less than 10.
  }
\end{lstlisting}
The same property does not apply to the \dlmc{||} operator due to short-circuit evaluation.
However negation of a \dlmc{@?} check is possible, and will allow for calling the function in the else branch of a conditional statement.

Using the ternary operator we can specify default values for functions that cannot be called.
Here negation is also supported.

\begin{lstlisting}[style=dlmc]
  int a = @?foo ? foo() : -1; // Valid
  int b = !@?foo ? -1 : foo(); // Valid, but somewhat confusing
\end{lstlisting}

\subsection{Selecting a policy}\label{time:authority}
As described in \cref{time:expressiveness}, a time policy can be associated with principal or it can be the default policy and thus apply to all other principals.
A single label could specify multiple time policies applying to various principals.
An example of such a set of policies is given below:
\begin{lstlisting}[style=dlmc]
  {{u->u, ec @ u:5m*10; 1h}}
\end{lstlisting}
Here the principal \dlmc{u} is allowed to access ten times in five minutes.
All other principals will only be allowed access once an hour.

To determine which policy applies in a given context we employ the effective authority, as described in \cref{dlm:auth_and_declass}.
Thus for the above policies, a programmer will have to raise the effective authority if he wants to employ \dlmc{u}s time policy.
The employed authority will be the least restrictive one available, given the effective authority when the function is called.
As described in \cref{time:restriction} there are multiple ways to define which time label is the most restrictive.
We do not provide means for declaring which type should be used by the tool.
Such a check is only required for the runtime aspects of time policies and is therefore ignored in the context of our tool.

In the example below we demonstrate how authority can be used to determine if and how a function should be called.

\begin{minipage}{\linewidth}
\begin{example}{Using authority for time policies}
In the following code example we make use of all the constructs specific to time policies.
We allow the program to busy-wait if we are allowed to act for the principal \dlmc{u}, otherwise we call the function \dlmc{foo} if we can do so without waiting.
\begin{lstlisting}[style=dlmc]
  int {{u->u, ec @ u:5m*10; 1h}} foo() {
    // Method body
  }
  
  int getfoo() {
    this -->? u { // Executed if authority can be raised
      return @foo(); // Will wait up to five minues before execution
    }
    else if (@?foo) { // Executed if the function can be called
      return foo();
    }
    else {
      return -1;
    }
  }
\end{lstlisting}
\end{example}
\end{minipage}

\mikkel{Time constructs applied to the examples}

As the tool developed only considers static evaluation we do not provide a run-time implementation of time policies or the effect of using the two constructs defined above.
In \cref{automata:timepolicies} we describe how timed automata can be constructed from our time policies.
