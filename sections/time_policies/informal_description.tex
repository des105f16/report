% !TEX root = ../../master.tex

\section{Informal Description}
% extend concept of output channels with ensuring that time policies are followed
% current authority dictates whether a function call to a time-labeled value can be made or not
% simple, yet expressive, time labels attached to readers
% in order to check compile time:
% - special constructs @ and @?
% - limited to return labels and function root scope variables (if time-labeled and not handled in the function)
Extending on the concept of controlling how information should exit the system, as we did previously using channels, we want to further this by adding time constraints.
As we have different principals in our system, which have different requirements/rights, we also want to be able to express this in our time policies.
We therefore suggest an extension of our previously defined labels, with the addition of time policies.
These time policies will extend upon the \emph{policy} labels, with the possibility to declare a time policy for each principal.
This is a way of saying that a specific owner of some data will only allow a certain principal to read his data while following some time constraint.

The \dlmc{@} symbol will be associated with anything time-related.
It is how the policies will have its principals extended with time policies, by ``attaching'' a principal with a \dlmc{@} it signifies a time policy for that principal.
If the owner of a policy has an attached time policy, it is meant to apply for all its readers.
The \dlmc{@} symbol will also be used for the two time constructs \dlmc{@?} (\emph{if-can-call})  and \dlmc{@} (\emph{await}), explained later in this section.

\subsection{Expressiveness}
We want our time policies to be expressive, so that we can cover a wide variety of cases.
We therefore introduce three types of time constituents, where different combinations of these will form a time policy.

This will be formally explained in \cref{time:pic}, however, we will give a small description and some simple examples of its usage here.
We have the three constituents \emph{period}, \emph{interval}, and \emph{count}.

\myparagraph{Period}
Period is an time-period with a start- and end-hour, where only within this period it is possible to access the data/function.
E.g. \dlmc{\{\{u->u, ec@09:00-10:00\}\}} signifies that \dlmc{ec} is only allowed to access when the time is between 09:00 and 10:00 in the morning.

\myparagraph{Interval}
Interval is a constant value indicating the minimum amount of time there should pass between each access.
\dlmc{\{\{u->u, ec@10m\}\}} requires a minimum of 10 minutes between each access.

\myparagraph{Count}
Count is in itself without meaning, as it says how many consecutive accesses within the given period and/or interval are allowed.
However, due to default values some meaning may apply to a case such as \dlmc{\{\{u->u, ec@*5\}\}}, where in this case 5 consecutive accesses are allowed within a day (24-hour period).

\myparagraph{Combined time policies}
It is possible to combine two or all three of the time constituents.

E.g. \dlmc{\{\{pc->u@10m*3\}\}}, indicating that the user is only allowed to access the user database thrice per 10 minutes.
Another way of explaining it; after the initial read another two reads \emph{can} follow, either way it is reset 10 minutes after the initial read.

Another example \dlmc{\{\{u->u, ec@00:00-09:00 30m\}\}}, where \dlmc{u} can access without limitations.
\dlmc{ec}, however, is restricted to reading between midnight and 9 in the morning, and with a minimum of 30 minutes passing between each access.

\subsection{Time constructs}
We have the two time constructs:

\begin{itemize}
  \item if-can-call \dlmc{@?}
  \item await \dlmc{@}
\end{itemize}

The two constructs represent run-time checks.
\dlmc{@?} takes one argument: a function identifier, and returns a boolean whether the function is callable or not, as per the time constraints.
\dlmc{@} is placed before a function call, forcing the program to busy-wait until the function is callable.

The constructs themselves currently have no run-time-related implementation.
However, they are used during compile time checking to ensure that whenever we access data with a time policy, that we have used one of these to somehow handle the time policy.

\subsection{Applications}
