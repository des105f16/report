% !TEX root = ../master.tex

\section{Informal Description}

\subsection{Scope}
As we have chosen a practical approach in developing \thelang~ we have had to apply a limited scope.
We have tried to include as many basic necessities as possible, in order to properly ascertain the usefulness of \thelang.

In the following is a crude list with what is currently supported in \thelang.
Following the list will be some elaborations where deemed necessary.
\begin{itemize}
  \item Variables
  \item Simple types (int, bool, char, string)
  \item Structs and arrays
  \item Declaring and calling functions
  \item Boolean and arithmetic operators
  \item Control structures
  \item External libraries
\end{itemize}

\subsubsection{Structs and arrays}
It is possible to define structs consisting of simple type fields.
However, it is not possible to define fields of other struct types -- i.e. nested structs.

Arrays are limited to simple types and structs.
Arrays are referred to as a pointer referenc  e, and arrays alone are referred to through pointers (more about this below).

\subsubsection{Boolean and arithmetic operators}
The following boolean and arithmetic operators are allowed:
\[ \tk +, \tk - \text{(unary and binary)}, \tk *, \tk /, \tk{\&\&}, \tk{||}, \tk{!} \]

Operator precedence has no say for our compiler, as we are only interested in what operands are used in expressions.
The actual operator precedence checks are only used for, and therefore handled by, the actual C compiler.

We have chosen to leave out the \tk{++} and \tk{-{}-} post-/pre-fix operators.

\subsubsection{Control structures}
It is possible to use \emph{if}-, \emph{if-else}-, and \emph{while}-statements.
We have, however, chosen to leave out the \emph{for}-statement, as it is can as easily be constructed as a while-statement.

\subsubsection{Pointers}
Pointers are completely left out of \thelang, as there are many troubles associated with assuring safety when data can be accessed through pointers.
\mikaelin{This sounds like something that needs further explaining. :)}
The pointer-syntax is still being used for arrays.

\subsection{Applying labels}
The first big thing in \thelang~ is of course applying labels to values and functions.
Labels are defined between the type and the identifier of a variable, or between the return type and identifier of a function.

Unlike types, labels can be omitted completely, which has two applications.
Firstly, it is possible to take old code and incrementally apply labels.
Secondly, it can be used to apply labels only where completely necessary, and relying on inference for the rest, thus limiting the clutter of labels throughout a program.
In the following examples, which are extensions on the C examples given in \cref{examples:sec}, we have explicitly declared labels for all variables, functions, and parameters.
Some of these labels have been deliberately faded to show labels which could be omitted, in order to display which labels are inferable.

The following subsections will describe the label constructions implemented by \thelang.
It will follow the \dlmc{check_password} example from \cref{example:sec:check_password}.
The fully updated source can be seen in \cref{example:code:check_password-explicit}.
The labelled version of \dlmc{calculate_bill} can be seen in \cref{example:code:calculate_bill-explicit}, however, it will not be described.

% !TEX root = ../../master.tex

\begin{lstlisting}[float, style=dlmc, numbers=left, caption={Labelled password checker example}, label=example:code:check_password-explicit]
#include <stdbool.h>
#include <string.h>

principal u, pc;

typedef struct user_info {
  char username[20];
  char password[20];
} user_info;

user_info {{u->}} get_login();
user_info {{pc->}} *get_users();
void {{output->u}} send_response(bool is_match);

bool {{u->u}} check_password(char {{u->}} *username,
    char {{u->}} *password) {
  int (*@\fadedlbl{pc->}@*) user_count = 100;
  user_info (*@\fadedlbl{pc->}@*) *users = get_users();
  int (*@\fadedlbl{pc->}@*) i = 0;
  bool (*@\fadedlbl{u->;pc->}@*) match = false;

  while (i < user_count) {
    if (!strcmp(users[i].username, username) &&
        !strcmp(users[i].password, password)) {
      match = true;
    }
    i = i + 1;
  }

  this -->? u, pc {
    return <|match, (*@\fadedlbl{u->u}@*)|>;
  }
}

int main(int argc, char **argv) {
  user_info (*@\fadedlbl{u->}@*) login = get_login();
  bool (*@\fadedlbl{u->u}@*) is_match = check_password(login.username,
      login.password);
  send_response(is_match);
}
\end{lstlisting}


\begin{lstlisting}[float, style=dlmc, numbers=left, caption={Labelled smart meter bill calculation example}, label=example:code:calculate_bill-explicit]
principal u, ec, s, output;

typedef struct usage {
  int start_time;
  int usage_in_Wh;
} usage;

typedef struct price {
  int start_time;
  int price_in_cents;
} price;

usage {{u->}} *get_latest_usage(){}
price {{_}} *get_latest_prices(){}
void {{output->u}} send_to_consumer(int bill){}
void {{output->ec}} send_to_electrical_company(int bill){}

int {{u->u, ec}} calculate_bill() {
  int {{s->}} usage_count = 100;
  int {{s->}} prices_count = 100;
  usage {{u->}} *latest_usage = get_latest_usage();
  price {{s->}} *latest_prices = get_latest_prices();
  int {{u->; s->}} result = 0;

  int {{s->}} i = 0;
  int {{u->; s->}} j = 0;
  while (i < usage_count) {
    while ((j < prices_count-1) && (latest_prices[j+1].start_time <= latest_usage[i].start_time)) {
      j = j + 1;
    }
    result = result + latest_usage[i].usage_in_Wh * latest_prices[j].price_in_cents;
    i = i + 1;
  }
  this -->? u, s {
    return <|result, {{u->u, ec}}|>;
  }
}

int main(int argc, char **argv) {
  int {{u->u, ec}} bill = calculate_bill();
  send_to_consumer(bill);
  send_to_electrical_company(bill);
}
\end{lstlisting}

