% !TEX root = ../master.tex
\section{Semantics}

\subsection{Semantics for determining labels}
The label environment is called $\iLbl$ and is defined by:
\[
  \iLbl = \iVar \cup \iFun \cup \ia \cup \ib \cup \ifb \rightharpoonup \mathbb{L}
\]

\noindent where $\ia$ is the authority label, $\ib$ the block label, and $\ifb$ the function block label.

\noindent $\mathbb{L}$ is the set of all possible labels, with the six types of labels: \\

\begin{tabular}{rl}
  \textbf{Policy labels}    & Like those described earlier \\
  \textbf{Variable labels}  & Denoted $\vlbl{x}$ \\
  \textbf{Constant labels}  & Denoted $\clbl{p}$ \\
  \textbf{Lower bound}      & Denoted $\bot$ \\
  \textbf{Upper bound}      & Denoted $\top$ \\
  \textbf{Join/meet labels} & Combinations of the above-mentioned labels
\end{tabular}

\paragraph{}\noindent
The function parameters environment is called $\iFunPar$ and is defined by:
\[
  %\iFunPar = \iFun \rightharpoonup \mathbf{P}(\iVar \times \mathbb{L})
  \iFunPar = \iFun \rightharpoonup (\iVar \times \mathbb{L})_1 \cup (\iVar \times \mathbb{L})_2 \cup \dots \cup (\iVar \times \mathbb{L})_n
\]

\paragraph{}\noindent
$\iCstr$ is a set of constraints where each $c \in \iCstr$ is on the form:
\[ l_1 \sqsubseteq l_2 \; | \; l_1, l_2 \in \mathbb{L} \]


\subsubsection{Transision systems}
$(\Gamma_P, \rightarrow_P, T_P)$, with
\begin{align*}
  \Gamma_P  & = (\iProg \times \iLbl \times \iFun) \\
            & \cup (\iDecv \times \iLbl \times \iFun) \\
            & \cup (\iDecf \times \iLbl \times \iFun) \\
            & \cup (\iLbl \times \iCstr \times \iFun) \\
  T_P       & = (\iLbl \times \iCstr \times \iFun)
\end{align*}

\paragraph{}\noindent
$(\Gamma_S, \rightarrow_S, T_S)$, with
\begin{align*}
  \Gamma_S & = (\iStm \times \iLbl) \cup (\iLbl \times \iCstr) \\
  T_S & = (\iLbl \times \iCstr)
\end{align*}

\noindent
Where rules are on the form:
\[ fun \vdash \argpart{S, lbl}{S}{(lbl, cstr)} \]

\subsubsection{Program}

\ctrule {PROG-COMP}
        {\argpart{P_1 \; P_2, lbl, fun}{P}{(lbl_2, cstr_1 \cup cstr_2, fun_2)}}
        {\argpart{P_1, lbl, fun}{P}{(lbl_1, cstr_1, fun_1)} \quad
          \argpart{P_2, lbl_1, fun_1}{P}{(lbl_2, cstr_2, fun_2)}}
        {}
\begin{trules}
\ctrule {PROG-VAR-DECL}
        {\argpart {D_V} {P} {(lbl_1, cstr, fun)} }
        {fun \vdash \argpart{D_V, lbl}{S}{(lbl_1, cstr)}}
        {}
\end{trules}

\subsubsection{Function declarations}

\ctrule {FUNC-DECL}
        {\argpart{T_f \, l_f \, f \tk ( T_1 \, l_1 \, x_1 \tk , T_2 \, l_2 \, x_2 \tk , \dots \tk , T_n \, l_n \, x_n \tk ) S, lbl, fun}{P}{(lbl_1, cstr, fun_1)}}
        {\argpart{fun_1 \vdash S, lbl_2}{S}{(lbl_3, cstr)}}
        {where $l_{x_i} = \begin{cases}
          \clbl{x_i} & \quad \text{if } l_i = \varepsilon \\
          l_i & \quad otherwise
        \end{cases}\quad \text{for } 1 \leq i \leq n$ \\
          and $l_f' = \begin{cases}
            \mathlarger\sqcupl_{i = 1}^n l_{x_i} & \quad \text{if } l_f = \varepsilon \\
            l_f & \quad \text{otherwise}
        \end{cases}$ \\
          and $lbl_1 = lbl[f \mapsto l_f']$ \\
          and $lbl_2 = lbl_1[x_1 \mapsto l_{x_1}, x_2 \mapsto l_{x_2}, \dots, x_n \mapsto l_{x_n}, \ifb \mapsto l_f']$ \\
          and $fun_1 = fun[f \mapsto ((l_1, x_1), (l_2, x_2), \dots, (l_n, x_n))]$}

\subsubsection{Statements}

\begin{trules}
  \ctrule{VAR-DECL-EXPR}
        {fun \vdash \argpart {T \, l \, x \tk = e, lbl} {S} {(lbl[x \mapsto l_x], cstr \cup \{ c \})} }
        {lbl, fun \vdash e \rightarrow_e (cstr, l_e)}
        {where $l_x = \begin{cases}
            \vlbl x & \quad \text{if } l = \varepsilon \\
            l & \quad \text{otherwise}
          \end{cases}$ \\
          and $l_\ib = lbl \, \ib$ \\
          and $c = " \, l_e \sqcup l_\ib \sqsubseteq l_x \, "$ }
  \trule{VAR-DECL}
        {fun \vdash \argpart {T \, l \, x, lbl} {S} {(lbl[x \mapsto l_x], cstr \cup \{ c \})} }
        {where $l_x = \begin{cases}
            \vlbl x & \quad \text{if } l = \varepsilon \\
            l & \quad \text{otherwise}
          \end{cases}$ \\
          and $l_\ib = lbl \, \ib$ \\
          and $c = "\bot \sqcup l_\ib \sqsubseteq l_x"$ }
\end{trules}

  \trule{EMPTY}
        {fun \vdash \argpart{\varepsilon, lbl}{S}{(lbl, \emptyset)}}
        {}
  \ctrule{EXPR}
        {fun \vdash \argpart{e, lbl}{S}{(lbl, cstr)}}
        {lbl, fun \vdash e \rightarrow_e (cstr, l_e)}
        {}
  \ctrule{COMP}
        {fun \vdash \argpart{S_1 \; S_2, lbl}{S}{(lbl_2, cstr_1 \cup cstr_2)}}
        {fun \vdash \argpart{S_1, lbl}{S}{(lbl_1, cstr_1)} \quad
          fun \vdash \argpart{S_2, lbl_1}{S}{(lbl_2, cstr_2)}}
        {}
  \ctrule{WHILE}
        {fun \vdash \argpart{\tk{while} \tk ( e \tk ) S, lbl}{S}{(lbl, cstr_1 \cup cstr_2 \cup \{ c \})}}
        {lbl_1, fun \vdash e \rightarrow_e (cstr_1, l_e) \quad
          fun \vdash \argpart{S, lbl_1}{S}{(lbl_2, cstr_2)}}
        {where $l_\ib = lbl \, \ib$ \\
          and $l_w = next$ \\
          and $lbl_1 = lbl[\ib \mapsto l_w]$ \\
          and $c = " \, l_e \sqcup l_\ib \sqsubseteq l_w \, "$}
  \ctrule{IF}
        {fun \vdash \argpart{\tk{if} \tk ( e \tk ) S, lbl}{S}{(lbl, cstr_1 \cup cstr_2 \cup \{ c \})}}
        {lbl, fun \vdash e \rightarrow_e (cstr_1, l_e) \quad
          fun \vdash \argpart{S, lbl[\ib \mapsto l_{if}]}{S}{(lbl_1, cstr_2)}}
        {where $l_\ib = lbl \, \ib$ \\
          and $l_{if} = next$ \\
          and $c = " \, l_e \sqcup l_\ib \sqsubseteq l_{if} \, "$}
  \ctrule{IF-ELSE}
        {fun \vdash \argpart{\tk{if} \tk ( e \tk ) S_1 \tk{else} S_2, lbl}{S}{(lbl, cstr_1 \cup cstr_2 \cup cstr_3 \cup \{ c \})}}
        {lbl, fun \vdash e \rightarrow_e (cstr_1, l_e) \quad
          fun \vdash \argpart{S_1, lbl_1}{S}{(lbl_2, cstr_2)} \quad
          fun \vdash \argpart{S_2, lbl_1}{S}{(lbl_3, cstr_3)}}
          {where $l_\ib = lbl \, \ib$ \\
            and $l_{if} = next$ \\
            and $lbl_1 = lbl[\ib \mapsto l_{if}]$ \\
            and $c = " \, l_e \sqcup l_\ib \sqsubseteq l_{if} \, "$}
  \ctrule{ASS}
        {fun \vdash \argpart{x \tk = e, lbl}{S}{(lbl, cstr \cup \{ c \})}}
        {lbl, fun \vdash e \rightarrow_e (cstr, l_e)}
        {where $l_\ib = lbl \, \ib$ \\
          and $c = " \, l_e \sqcup l_\ib \sqsubseteq l_x \, "$ }
  \ctrule{RET-EXPR}
        {fun \vdash \argpart{\tk{return} e, lbl}{S}{(lbl, cstr \cup \{ c \})}}
        {lbl, fun \vdash e \rightarrow_e (cstr, l_e)}
        {where $l_\ib = lbl \, \ib$ \\
          and $l_\ifb = lbl \, \ifb$ \\
          and $c = " \, l_e \sqcup l_\ib \sqsubseteq l_\ifb \, "$}
  \trule{RET}
        {fun \vdash \argpart{\tk{return}, lbl}{S}{(lbl, \emptyset)}}
        {}
  \ctrule{ACTS-THIS}
        {fun \vdash \argpart{\tk{this} \tk{-->?} p_1 \tk , p_2 \tk , \dots \tk , p_n \; S, lbl}{S}{(lbl, cstr)}}
        {fun \vdash \argpart{S, lbl[\ia \mapsto l_\ia']}{S}{(lbl_1, cstr)}}
        {where $l_\ia = lbl \, \ia$ \\
          and $l_\ia' = l_\ia \sqcup \mathlarger\sqcupl_{i=1}^n \, \{ p_i \rightarrow \emptyset \}$}
    \ctrule{ACTS-CALLER}
        {fun \vdash \argpart{\tk{caller} \tk{-->?} p_1 \tk , p_2 \tk , \dots \tk , p_n \; S, lbl}{S}{(lbl, cstr)}}
        {fun \vdash \argpart{S, lbl[\ia \mapsto l_\ia']}{S}{(lbl_1, cstr)}}
        {where $l_\ia = lbl \, \ia$ \\
          and $l_\ia' = l_\ia \sqcup \mathlarger\sqcupl_{i=1}^n \, \{ p_i \rightarrow \emptyset \}$}

\subsection{Semantics for deriving labels for expressions}
$(\Gamma_e, \rightarrow_e, T_e)$, with
\begin{align*}
  \Gamma_e =  & \;      \iExp \cup (\iCstr \times \mathbb{L}) \\
  T_e =       & \; (\iCstr \times \mathbb{L})
\end{align*}
Where transitions are of the form $lbl, fun \vdash e \rightarrow_e (cstr, l)$.

\subsubsection{Declassification}
\begin{trules}
  \ctrule{DEC}
        {lbl, fun \vdash \tk{<|} e \tk{|>} \rightarrow_e (cstr \cup \{ c \}, l_d)}
        {lbl, fun \vdash e \rightarrow_e (cstr, l_e)}
        {where $l_\ia = lbl \, \ia$ \\
          and $l_{d} = next$ \\
          and $c = " \, l_e \sqsubseteq l_d \sqcup l_\ia \, "$}
  \ctrule{DEC-EXPL}
        {lbl, fun \vdash \tk{<|} e \tk , l \tk{|>} \rightarrow_e (cstr \cup \{ c \}, l)}
        {lbl, fun \vdash e \rightarrow_e (cstr, l_e)}
        {where $l_\ia = lbl \, \ia$ \\
          and $l_d = next$ \\
          and $c = " \, l_e \sqsubseteq l \sqcup l_\ia \, "$}
\end{trules}

\subsubsection{Function call}
\begin{trules}
  \ctrule {FUN-CALL-LIB}
          {lbl, fun \vdash f \tk ( e_1 \tk , e_2 \tk , \dots \tk , e_n \tk ) \rightarrow_e (cstr', l')}
          {lbl, fun \vdash e_1 \rightarrow_e (cstr_1, l_{e_1}) \quad e_2 \rightarrow_e (cstr_2, l_{e_2}) \quad
            \dots \quad e_n \rightarrow_e (cstr_n, l_{e_n})}
          {if $lbl \, f = \emptyset$ \\
            and $l' = \mathlarger\sqcupl_{i=1}^n l_{e_i}$ \\
            and $cstr' = \bigcup\limits_{i=1}^n cstr_i$}
\end{trules}

\subsubsection{Arithmetics}
\begin{trules}
  \trule  {VAR}
          {lbl, fun \vdash x \rightarrow_e (\emptyset, l)}
          {where $lbl \, x = l$}
  \trule  {CONST}
          {lbl, fun \vdash c \rightarrow_e (\emptyset, \bot)}
          {}
  \ctrule {BINARY-OP}
          {lbl, fun \vdash e_1 \; op_b \; e_2 \rightarrow_e (cstr_1 \cup cstr_2, l_1 \sqcup l_2)}
          {lbl, fun \vdash e_1 \rightarrow_e (cstr_1, l_1) \;\; lbl, fun \vdash e_2 \rightarrow_e (cstr_2, l_2)}
          {}
  \ctrule {UNARY-OP}
          {lbl, fun \vdash op_u \; e \rightarrow_e (cstr, l)}
          {lbl, fun \vdash e \rightarrow_e (cstr, l)}
          {}
  \ctrule {PAREN}
          {lbl, fun \vdash \tk ( e \tk ) \rightarrow_e (cstr, l)}
          {lbl, fun \vdash e \rightarrow_e (cstr, l)}
          {}
\end{trules}
