% !TEX root = ../master.tex

\section{Timed Automata}
Here we will shortly define and describe timed automata, based on \cite{timed_automata}.

The timed automata is an extension to the $\omega$-automata formalism, allowing for the manipulation of clock-variables, as well as clock-based restrictions.

\begin{definition}{Timed automaton}
A timed automaton is a tuple $(\Sigma, S, S_0, C, E)$, where
\begin{itemize}
  \item $\Sigma$ is a finite alphabet,
  \item $S$ is a finite set of states,
  \item $S_0 \subseteq S$ is a set of start states,
  \item $C$ is a finite set of clocks, and
  \item $E \subseteq S \times S \times \Sigma \times 2^C \times \phi(C)$ gives the set of transitions. \\
    An edge $(s, s', \sigma, \lambda, \delta)$ represents a transition from state $s$ to state $s'$ on input symbol $\sigma$.
    The set $\lambda \subseteq C$ gives the clocks to be reset with this transition, and $\delta$ is a clock constraint over C.
\end{itemize}
\end{definition}
