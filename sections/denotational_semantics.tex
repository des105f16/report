% !TEX root = ../master.tex

\newcommand{\iP}{\mathcal{P}}
\newcommand{\iD}{\mathcal{D}}
\newcommand{\iV}{\mathcal{V}}
\newcommand{\iF}{\mathcal{F}}
\newcommand{\iS}{\mathcal{S}}
\newcommand{\iE}{\mathcal{E}}
\newcommand{\iL}{\mathcal{L}}
\newcommand{\iO}{\mathcal{O}}

\newcommand{\iLV}{LV}

\newcommand{\dblSq}[1]{[\![ #1 ]\!]}

\newcommand{\rspace}{\\[1em]}
\newcommand{\dsWhere}[1]{\quad \text{ where } #1}
\newcommand{\dsAnd}[1]{\quad \text{ and } #1}
\newcommand{\dsIf}[1]{\quad \text{ if } #1}

\section{Semantics}

\subsection{Semantic Domains}
In the following, $\mathbb P$ denotes the power set.
\begin{align*}
\iLV      & = v(\iVar) \cup c(\iVar) \cup p(\mathbb{L}) \cup j(\iLV \times \iLV) \\
\iEnvL  & = \iVar \cup \{ \ia, \ib \} \rightharpoonup \iLV \\
\iEnvP  & = \iFun \cup \{ \ifb \} \rightarrow \iLV \times \mathbb P (\iVar \times \iLV) \\
\iCstr  & = \mathbb P (\iLV \times \iLV)
\end{align*}

$\iLV$~ is a set of \emph{Label Values}, which contains tagged values:
\begin{description}
  \item[$v$] -- Variable labels with variable identifier
  \item[$c$] -- Constant labels with variable identifier
  \item[$p$] -- Policy labels, as those described in \cref{dlm}
  \item[$j$] -- Join labels, the join of two label values
\end{description}

\subsection{Semantic Functions}
\begin{align*}
\iP & : \iProg \rightarrow \iCstr \\
\iD & : \iDec \rightarrow \iCstr \times \iEnvL \times \iEnvP \\
\iV & : \iDecv \rightarrow \iCstr \times \iEnvL \\
\iF & : \iDecf \rightarrow \iCstr \times \iEnvL \times \iEnvP \\
\iS & : \iStm \rightarrow \iCstr \times \iEnvL \\
\iE & : \iExp \rightarrow \iCstr \times \iLV \\
\iL & : \iLbl \cup \iPol \rightarrow \iLV \cup \{ \varepsilon \}
\end{align*}

\subsection{Semantic Equations}

\subsubsection{Program and Declaration}
\begin{align*}
& \iP\dblSq{D} = \icstr \dsWhere{\iD\dblSq{D} \; empty\icstr \; empty\ienvL \; empty\ienvP = (\icstr, \ienvL, \ienvP)} \rspace
& \iD\dblSq{D_V} \; \icstr \; \ienvL \; \ienvP = (\icstr_2, \ienvL_2, \ienvP) \dsWhere{\iV\dblSq{D_V} \; \icstr \; \ienvL \; \ienvP = (\icstr_2, \ienvL_2)} \rspace
& \iD\dblSq{D_F} \; \icstr \; \ienvL \; \ienvP = \iF\dblSq{D_F} \; \icstr \; \ienvL \; \ienvP \rspace
& \iD\dblSq{D_1 \; D_2} \; \icstr \; \ienvL \; \ienvP = (\icstr_3, \ienvL_3, \ienvP_3) \\
& \quad \text{ where } \iD\dblSq{D_2} \; \icstr_2 \; \ienvL_2 \; \ienvP_2  = (\icstr_3, \ienvL_3, \ienvP_3) \\
& \quad \text{ and } \iD\dblSq{D_1} \; \icstr \; \ienvL \; \ienvP = (\icstr_2, \ienvL_2, \ienvP_2) \\
& \textit{Alternatively: } \iD\dblSq{D_1 \; D_2} = \iD\dblSq{D_2} \circ \iD\dblSq{D_1}
\end{align*}

\subsubsection{Label}
\begin{align*}
& \iL\dblSq{\varepsilon} \; \ienvL = \varepsilon \rspace
%
& \iL\dblSq{\tk{\{\{} pol \tk{\}\}}} \; \ienvL = \iL\dblSq{pol} \; \ienvL \rspace
%
& \iL\dblSq{pol_1 \tk ; pol_2} \; \ienvL = j(\iLV_1, \iLV_2) \\
& \dsWhere{\iL\dblSq{pol_1} \; \ienvL = \iLV_1} \\
& \dsAnd{\iL\dblSq{pol_2} \; \ienvL = \iLV_2} \rspace
%
& \iL\dblSq{x} \; \ienvL = \ienvL x \text{\mikaelin{Is this right?}} \rspace
%
& \iL\dblSq{p_0 \tk{->} p_1 \tk , p_2 \tk, \dots \tk , p_k} \; \ienvL = p(\{p_0' \rightarrow p_1', p_2', \dots, p_k'\}) \\
& \dsWhere{p_i' = principal(p_i) \text{, for } 0 \leq i \leq k} \rspace
%
& \iL\dblSq{\tk \_} \; \ienvL = p(\bot) \rspace
& \iL\dblSq{\tk{\^{}}} \; \ienvL = p(\top)
\end{align*}

\subsubsection{Variable Declaration}
\begin{align*}
& \iV\dblSq{T \; l \; x \tk = e} \; \icstr \; \ienvL \; \ienvP = (\icstr \cup \icstr_2 \cup \{ c \}, \ienvL[x \mapsto L_x]) \\
& \dsWhere{\iL\dblSq{l} \; \ienvL = L} \\
& \dsAnd{\iE\dblSq{e} \; \ienvL \; \ienvP = (\icstr_2, L_e)} \\
& \dsAnd{L_x = \begin{cases}
    v(x) & \quad \text{if } L = \varepsilon \\
    L & \quad \text{otherwise}
  \end{cases}} \\
& \dsAnd{L_\ib = \ienvL \, \ib} \\
& \dsAnd{c = " \, L_e \sqcup L_\ib \sqsubseteq L_x \, "} \rspace
& \iV\dblSq{T \; l \; x} \; \icstr \; \ienvL \; \ienvP = (cstr \cup \{ c \}, \ienvL[x \mapsto L_x])\\
& \dsWhere{\iL\dblSq{l} \; \ienvL = L} \\
& \dsAnd{L_x = \begin{cases}
    v(x) & \quad \text{if } L = \varepsilon \\
    L & \quad \text{otherwise}
  \end{cases}} \\
& \dsAnd{L_\ib = \ienvL \, \ib} \\
& \dsAnd{c = " \, \bot \sqcup L_\ib \sqsubseteq L_x \, "}
\end{align*}

\subsubsection{Function Declarations}
\begin{align*}
& \iF\dblSq{T_f \, l_f \, f \tk ( T_1 \, l_1 \, x_1 \tk , T_2 \, l_2 \, x_2 \tk , \dots \tk , T_n \, l_n \, x_n \tk ) S} \; \ienvL \; \ienvP
  =  (\icstr, \ienvL, \ienvP_2) \\
& \dsWhere{\iS\dblSq{S} \; \ienvL_2 \; \ienvP_2 = (\icstr, \ienvL_3)} \\
& \dsAnd{\ienvL_2 = \ienvL[x_1 \mapsto LV_1', x_2 \mapsto LV_2', \dots, x_n \mapsto LV_n']} \\
& \dsAnd{\ienvP_2 = \ienvP[\ifb \mapsto (LV_f', \emptyset), f \mapsto (LV_f', \{(LV_1', x_1), (LV_2', x_2), \dots, (LV_n', x_n)\})]} \\
& \dsAnd{LV_f' = \begin{cases}
    \mathlarger\sqcupl_{i = 1}^n LV_i & \quad \text{if } LV_f = \varepsilon \\
    LV_f & \quad \text{otherwise}
  \end{cases}} \\
& \dsAnd{\iL\dblSq{l_f} \; \ienvL = LV_f} \\
& \dsAnd{LV_i' = \begin{cases}
    c(x_i) & \quad \text{if } LV_i = \varepsilon \\
    LV_i & \quad otherwise
  \end{cases}} \\
& \dsAnd{\iL\dblSq{l_i} \; \ienvL = LV_i \text{ for all } 0 \leq i \leq n}
\end{align*}
