% !TEX root = ../master.tex

\newcommand{\iP}{\mathcal{P}}
\newcommand{\iD}{\mathcal{D}}
\newcommand{\iV}{\mathcal{V}}
\newcommand{\iF}{\mathcal{F}}
\newcommand{\iS}{\mathcal{S}}
\newcommand{\iE}{\mathcal{E}}
\newcommand{\iL}{\mathcal{L}}
\newcommand{\iO}{\mathcal{O}}

\newcommand{\iLV}{LV}

\newcommand{\dblSq}[1]{[\![ #1 ]\!]}

\newcommand{\rspace}{\\[0.5em]}
\newcommand{\dsWhere}[1]{\quad \text{ where } #1}
\newcommand{\dsAnd}[1]{\quad \text{ and } #1}

\section{Semantics}

\subsection{Abstract Syntax}

\subsubsection{Semantic Domains}
In the following, $\mathcal P$ denotes the power set.
\mikaelin{Not sure what to do with LV yet...}
\begin{align*}
\iLV      & = v(\iVar) \cup c(\iVar) \cup p(\mathbb{L}) \cup j(\iLV \times \iLV) \\
\iEnvL  & = \iVar \cup \iFun \cup \{ \ia, \ib, \ifb \} \rightharpoonup \iLV \\
\iEnvP  & = \iFun \rightarrow \mathcal{P}(\iVar \times \iLV) \\
\iCstr  & = \mathcal P (\iLV \times \iLV)
\end{align*}

\subsubsection{Semantic Functions}
\begin{align*}
\iP & : \iProg \rightarrow \iCstr \\
\iD & : \iDec \rightarrow \iCstr \times \iEnvL \times \iEnvP \\
\iV & : \iDecv \rightarrow \iCstr \times \iEnvL \\
\iF & : \iDecf \rightarrow \iCstr \times \iEnvL \times \iEnvP \\
\iS & : \iStm \rightarrow \iCstr \times \iEnvL \\
\iE & : \iExp \rightarrow \iCstr \times \iLV \\
\iL & : \iLbl \cup \iPol \rightarrow \iLV \cup \{ \varepsilon \}
\end{align*}

\subsubsection{Semantic Equations}
\begin{align*}
% PROGRAM
& \iP\dblSq{D} = \icstr \dsWhere{\iD\dblSq{D} \; empty\icstr \; empty\ienvL \; empty\ienvP = (\icstr, \ienvL, \ienvP)} \rspace
% DECLARATION
& \iD\dblSq{D_V} \; \icstr \; \ienvL \; \ienvP = (\icstr_2, \ienvL_2, \ienvP) \dsWhere{\iV\dblSq{D_V} \; \icstr \; \ienvL \; \ienvP = (\icstr_2, \ienvL_2)} \rspace
& \iD\dblSq{D_F} \; \icstr \; \ienvL \; \ienvP = \iF\dblSq{D_F} \; \icstr \; \ienvL \; \ienvP \rspace
& \iD\dblSq{D_1 \; D_2} \; \icstr \; \ienvL \; \ienvP = (\icstr_3, \ienvL_3, \ienvP_3) \\
& \quad \text{ where } \iD\dblSq{D_2} \; \icstr_2 \; \ienvL_2 \; \ienvP_2  = (\icstr_3, \ienvL_3, \ienvP_3) \\
& \quad \text{ and } \iD\dblSq{D_1} \; \icstr \; \ienvL \; \ienvP = (\icstr_2, \ienvL_2, \ienvP_2) \\
& \textit{Alternatively: } \iD\dblSq{D_1 \; D_2} = \iD\dblSq{D_2} \circ \iD\dblSq{D_1} \rspace
% VARIABLE DECLARATION
& \iV\dblSq{T \; l \; x \tk = e} \; \icstr \; \ienvL \; \ienvP = (\icstr \cup \icstr_2 \cup \{ c \}, \ienvL[x \mapsto L_x]) \\
& \dsWhere{\iL\dblSq{l} \; \ienvL = L} \\
& \dsAnd{\iE\dblSq{e} \; \ienvL \; \ienvP = (\icstr_2, L_e)} \\
& \dsAnd{L_x = \begin{cases}
    v(x) & \quad \text{if } L = \varepsilon \\
    L & \quad \text{otherwise}
  \end{cases}} \\
& \dsAnd{L_\ib = \ienvL \, \ib} \\
& \dsAnd{c = " \, L_e \sqcup L_\ib \sqsubseteq L_x \, "} \rspace
& \iV\dblSq{T \; l \; x} \; \icstr \; \ienvL \; \ienvP = (cstr \cup \{ c \}, \ienvL[x \mapsto L_x])\\
& \dsWhere{\iL\dblSq{l} \; \ienvL = L} \\
& \dsAnd{L_x = \begin{cases}
    v(x) & \quad \text{if } L = \varepsilon \\
    L & \quad \text{otherwise}
  \end{cases}} \\
& \dsAnd{L_\ib = \ienvL \, \ib} \\
& \dsAnd{c = " \, \bot \sqcup L_\ib \sqsubseteq L_x \, "} \rspace
\end{align*}
