% !TEX root = ../master.tex

In this chapter we will extend our previous description of \thelang{} with the use of time policies.
The same concepts as previous still hold, as we want a simple way of declaring time policies to certain values in a program, in order to restrict the access to these values.

First, we will introduce some minor theory about timed automata which is useful for the understanding of the following sections.
Then we will look at a current possible solution: ``The Timed Decentralized Label Model'', and argue why we have chosen to base our own solution on a more practical approach.
We will then make our own proposal of a practical extension of the previously discussed security labels, extended with time policies.
After both informal and formal descriptions of these new time policies, we will show how they can be translated into timed automatas, so that relevant theory may still apply.
